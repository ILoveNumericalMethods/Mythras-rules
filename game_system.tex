\documentclass[10pt]{article}
\usepackage[utf8]{inputenc}
\usepackage[margin=1in]{geometry}
\usepackage{booktabs}
\usepackage{longtable}
\usepackage{xcolor}
\usepackage{enumitem}


\usepackage[colorlinks=true,                % ссылки будут цветными, а не в рамке
            linkcolor=blue,                  % цвет внутренних ссылок (на главы, формулы и т.д.)
            urlcolor=blue,                    % цвет ссылок на интернет-страницы
            citecolor=blue,                    % цвет ссылок на библиографию
            pdfborder={0 0 0}                  % убирает рамку (на всякий случай)
           ]{hyperref}

% Custom commands for consistent formatting
\newcommand{\skill}[1]{\textbf{#1}}
\newcommand{\tableref}[1]{Table~\ref{#1}}
\newcommand{\secref}[1]{Section~\ref{#1}}
\newcommand{\pageRef}[1]{[→ p. #1]}

\title{Mythras: Game System \\ \large Reference Document (Action, Time, Movement \& Core Mechanics)}
\author{}
\date{}

\setlength{\parindent}{0pt}
\setlength{\parskip}{6pt}

\begin{document}

\maketitle

\tableofcontents
\newpage

\section{ACID}
\label{sec:acid}

Acids come from many different sources and are used extensively in alchemical research. All acids are  classified as either Weak, Strong or Concentrated. 

A splash or spray of acid lasts only for a few \textcolor{red}{Combat Rounds} before it loses potency. Immersion in a considerable volume of acid inflicts the damage every round until the victim or location is removed and treated.

For generic acids it is assumed that armour protects against the effects of acid but does not stop it, its Armour Points being reduced by the acid's damage until it reaches zero hit points, at which point the damage is then transferred to the hit location it protected. Armour reduced to zero Armour Points is considered useless as the acid burns through bindings, straps and joints.

\begin{table}[h]
\centering
\caption{Acid Table}
\label{tab:acid}
\begin{tabular}{lll}
\toprule
\textbf{Acid Type} & \textbf{Damage} & \textbf{Duration} \\
\midrule
Weak    & 1d2 & 1 Combat Round \\
Strong  & 1d4 & 1d2 Combat Rounds \\
Concentrated & 1d6 & 1d3 Combat Rounds \\
\bottomrule
\end{tabular}
\end{table}


\section{ACTION, TIME AND MOVEMENT}


\subsection{Traveling}
Typical distances that are covered during travel are listed below:

\begin{table}[h]
\centering
\caption{Time Travel Table}
\label{tab:strategic-travel}
\begin{tabular}{ll}
\toprule
\textbf{Travel Example} & \textbf{Distance Covered (km) per Day} \\
\midrule
Walking & 30 per Day \\
Horseback at casual speed & 60 per Day \\
Wagon at casual speed & 15 per Day \\
Open Sea, favourable conditions & 150-300 (24h) \\
Open Sea, unfavourable conditions & 0-60 (24h) \\
Coast or River, favourable & 30-60 per Day \\
Coast or River, unfavourable & 0-30 per Day \\
\bottomrule
\end{tabular}
\end{table}

If characters need to increase the distances given in the Time Travel Table, then they can effectively add half again to the distance travelled in the Strategic Time period but gain an enduring level of 
\hyperref[sec:fatigue]{Fatigue} as a result. Naturally an appropriate skill roll (Drive, Ride, Athletics, and so on) also needs to be made to successfully increase the distance covered.


\subsection{Movement}
\label{sec:movement}
Movement is broken down into three 'gaits' which are Walking, Running and Sprinting.
\begin{itemize}[leftmargin=*]
    \item \textbf{Walking} is the average speed a member of a particular species ambles along at when in no particular hurry. This is normally referred to as a creature's base Movement Rate.
    \item \textbf{Running} is a trot or jog, at a speed which can be maintained over long periods. Maximum running speed is triple Movement Rate, which may vary according to the bonus granted by Athletics skill \secref{sec:skills}.
    \item \textbf{Sprinting} is flat out movement at top speed, which can only be maintained for brief periods. Peak sprinting speed is five times Movement Rate, which again may be increased according to whatever bonus is granted from Athletics skill.
\end{itemize}



\begin{table}[h]
\centering
\caption{Comparative Movement Table (in meters)}
\label{tab:comparative-movement}
\begin{tabular}{lcccccc}
\toprule
\textbf{Time Period} & \textbf{Mov 4} & \textbf{Mov 6} & \textbf{Mov 8} & \textbf{Mov 10} & \textbf{Mov 12} & \textbf{Mov 14} \\
\midrule
Combat Round & 4m & 6m & 8m & 10m & 12m & 14m \\
Minute & 48m & 72m & 96m & 120m & 144m & 168m \\
Hour & 2.9km & 4.3km & 5.8km & 7.2km & 8.6km & 10.1km \\
\bottomrule
\end{tabular}
\end{table}

\subsection{Moving in Armour}
\label{sec:armor-movement}
Worn armour acts against character Movement Rates, and certain kinds of actions such as swimming or climbing. The armour's \textcolor{red}{Initiative Penalty} is applied to Movement of different kinds in the following ways:
\begin{itemize}[leftmargin=*]
    \item \textbf{Walking:} Armour does not interfere with walking movement, although it can increase the \hyperref[sec:fatigue]{Fatigue level}.
    \item \textbf{Running or Sprinting:} Subtract the Armour Penalty from the running and sprinting speed.
    \item \textbf{Swimming:} Take the character's \textcolor{red}{swimming speed}  divide by two (rounding up), and subtract the Armour Penalty. If the result is zero, the character cannot move, and barely keeps himself afloat. If the result is negative, then the character sinks.
    \item \textbf{Climbing rough surface:} Half the Armour Penalty (rounded up) is subtracted from the base Movement Rate.
    \item \textbf{Climbing a steep surface:} The Armour Penalty is subtracted from the base Movement Rate.
    \item \textbf{Climbing a sheer surface:} Double the Armour Penalty is subtracted from the base Movement Rate.
    \item \textbf{Jumping:} Reduce the distance in metres \textcolor{red}{the character can jump}  by half the Armour Penalty (rounded up). For standing jumps this impairment is halved.
\end{itemize}


\section{AGEING}
\label{sec:ageing}
All characters age, and with age come certain consequences. The signs of ageing start at Early Middle Age (40 years for humans). As a character passes into a new Ageing Band he must make both an Endurance roll and a Willpower roll at the grades noted. If a roll is failed then he experiences Ageing Effects as shown in the Ageing Effects table.

Each characteristic affected by ageing reduces by 1d3 points. If any characteristic is reduced to zero from ageing, the character dies due to his terminal frailty.

\begin{table}[h]
\centering
\caption{Age Bands}
\label{tab:age-bands}
\begin{tabular}{ll}
\toprule
\textbf{Age Band} & \textbf{Roll Grade} \\
\midrule
Early Middle Age (40-49) & Easy \\
Middle Age (50-59) & Standard \\
Late Middle Age (60-69) & Hard \\
Old Age (70-79) & Formidable \\
Advanced Old Age (80-89) & Herculean \\
Doddering (90+) & Hopeless \\
\bottomrule
\end{tabular}
\end{table}

\begin{table}[h]
\centering
\caption{Ageing Effects Table}
\label{tab:ageing-effects}
\begin{tabular}{cll}
\toprule
\textbf{1d6} & \textbf{Physical (Failed Endurance)} & \textbf{Mental (Failed Willpower)} \\
\midrule
1-2 & STR & INT \\
3-4 & CON & POW \\
5-6 & DEX & CHA \\
\bottomrule
\end{tabular}
\end{table}

\section{ASPHYXIATION, DROWNING AND SUFFOCATION}
\label{sec:asphyxiation}
Characters can hold their breath for a number of seconds equal to their Endurance skill. However the character must be prepared (filling the lungs with as much air as possible); if not, then the period is halved if the character was in a passive situation, or reduced to one fifth if the character was engaged in strenuous activity.

Once the period of held breath is over, characters must make an Endurance roll:
\begin{itemize}[leftmargin=*]
    \item If the roll is a critical success, no further deterioration occurs.
    \item If the roll is successful, the character accrues an extra level of \hyperref[sec:fatigue]{Fatigue}.
    \item If the roll fails, the character sustains 1d2 extra levels of \hyperref[sec:fatigue]{Fatigue} that round.
    \item If the roll is fumbled, the character sustains 1d3 extra levels of \hyperref[sec:fatigue]{Fatigue} that round.
\end{itemize}
Without aid, death from asphyxiation is usually swift. If the asphyxiation ends before the character dies, they recover Fatigue levels lost to suffocation relatively quickly; regaining one level per minute.


\section{CHARACTER IMPROVEMENT}
\label{sec:improvement}
The mechanism for character improvement is the Experience Roll. Games Masters dispense Experience Rolls at an appropriate juncture in the campaign. The number of Experience Rolls given in any one sitting should be between two and four.

\subsection{Increasing Existing Skills}
\label{sec:skill-improvement}
Any skill on the character sheet, Standard or Professional, can be increased by spending one Experience Roll.
\begin{itemize}[leftmargin=*]
    \item The player rolls 1d100 and compares it to the skill being increased. The character's INT is added to the roll.
    \item If the number rolled is equal to or greater than the skill being improved it increases by 1d4 + 1\%.
    \item If the number rolled is less than the skill selected, the skill still increases but only by 1\%.
    \item If a character fumbled any skill during the course of the preceding session(s), the fumbled skill gains a free increase of 1\%.
\end{itemize}

\subsection{Increasing Characteristics}
\label{sec:characteristic-improvement}
Characteristics can, like skills, be improved through Experience Rolls, which represent training regimes. Such increases are artificial boosts which normally atrophy after the training exercises cease.

To achieve and maintain characteristic increases requires that a character reduce his regular allotment of Experience Rolls by one or more points. Each Experience Roll sacrificed in this manner boosts the trained characteristic by one tenth of its rolled \textcolor{red}{species maximum}.

No matter how much training is undertaken, no characteristic can exceed its species maximum. SIZ is the exception to the above rules. It cannot be increased through mundane means.

\subsection{Increasing Passions}
\label{sec:passions-improvement}
As described under Passions, the value of a passion may be increased with Experience Rolls in exactly the same way as a skill \secref{sec:skill-improvement}.

\subsection{Training}
\label{sec:training}

Skills can be improved without expending Experience Rolls, through help of a mentor; either a trainer or a teacher. A trainer must have at least 20\% more than the character in the skill being trained, and the degree of improvement rests on the difference.

Characters must spend one full week in training to benefit from a training increase. At the end of the training period the skill being trained improves by the die roll indicated on the Training Chart, with any modifiers due to the Teaching skill.

\begin{table}[h]
\centering
\caption{Training Chart}
\label{tab:training}
\begin{tabular}{ll}
\toprule
\textbf{Difference} & \textbf{Improvement} \\
\midrule
Less than 20\% & 1d2 \\
21-30\% & 1d2 \\
31-40\% & 1d3 \\
41-50\% & 1d4 \\
51-60\% & 1d6 \\
61-70\% & 1d6+1 \\
71-80\% & 1d6+2 \\
81-90\% & 1d6+3 \\
91-100\% & 1d6+4 \\
Each 10\% & +1 \\
\bottomrule
\end{tabular}
\end{table}

\subsubsection{Teachers}
\label{sec:teachers}
A teacher is a professional trainer who possesses the Teach skill, using it to improve the quality of his tuition. They utilise the same Training Chart above, but modify the amount increased according to the result of a Teach roll:
\begin{itemize}[leftmargin=*]
    \item Critical Success: Skill improvement increases by two Training steps
    \item Success: Skill improvement increases by one Training step
    \item Failure: No effect on improvement
    \item Fumble: Skill improvement decreases by one Training step, which may result in the character gaining no improvement through the training
\end{itemize}

\section{DISEASE AND POISON}
\label{sec:disease-poison}
All diseases and poisons manifest a number of traits important to their effects.

\subsection{Disease and Poison Traits}
\begin{itemize}[leftmargin=*]
    \item \textbf{Application:} The method of how the disease or poison is introduced into the victim (Ingestion, Inhaled, Contact, Injected).
    \item \textbf{Potency:} The virulence of the disease or poison. This value is set against an appropriate resisting skill (usually Endurance or Willpower) in an opposed roll.
    \item \textbf{Resistance:} How the disease or poison is resisted - either Endurance or Willpower.
    \item \textbf{Onset Time:} The delay before the disease or poison takes effect.
    \item \textbf{Duration:} How long a disease or poison's Conditions last.
    \item \textbf{Conditions:} Every toxin has one or more Conditions with specific effects.
    \item \textbf{Antidote/Cure:} If the toxin can be treated it will be noted here.
\end{itemize}

\begin{longtable}{lp{10cm}}
\caption{Disease and Poison Conditions}
\label{tab:disease-conditions} \\
\toprule
\textbf{Condition} & \textbf{Effects} \\
\midrule
\endfirsthead
\multicolumn{2}{c}{\tablename\ \thetable\ -- continued from previous page} \\
\toprule
\textbf{Condition} & \textbf{Effects} \\
\midrule
\endhead
\bottomrule
\endfoot
Agony & Victim is hindered by intense pain. Any skill involving use of the affected area must also be less or equal to the character's Willpower, otherwise the attempt fails. \\
Asphyxiation & Victim suffers \hyperref[sec:asphyxiation]{asphyxiation}  – he collapses incapacitated, unable to breathe. \\
Bleeding & Victim suffers from either internal bleeding or surface haemorrhaging which leads to blood loss. \\
Blindness & Victim becomes blind. \\
Confusion & Victim cannot use any knowledge, communication or magic skill. \\
Contagious & Victim can transfer the poison or disease by touch. \\
Deafness & Victim loses his hearing. \\
Death & Victim collapses incapacitated, and dies after a number of rounds equal to his CON characteristic. \\
Dumbness & Victim's vocal chords are paralysed, preventing verbal communication. \\
Exhaustion & Victim gains an extra level of \hyperref[sec:fatigue]{Fatigue}. \\
Fever & Victim's body temperature fluctuates wildly. All skills suffer a difficulty grade of Hard. \\
Hallucinations & Victim experiences delusions and cannot differentiate between real and imaginary experiences. \\
Maiming & Victim suffers a permanent loss of 1 Hit Point in the location(s) affected. \\
Mania & Victim is driven to follow some compulsion. \\
Nausea & Victim cannot eat, and must roll against his Endurance every time he performs a stressful physical action to avoid being physically sick. \\
Paralysis & Victim is unable physically to move. \\
Sapping & Victim has their Magic Points reduced. \\
Unconscious & Victim loses consciousness for a period specified. \\
\end{longtable}

\subsection{Sample Diseases}
\label{sec:sample-diseases}
\textbf{Pyrohoxia}
\begin{itemize}[leftmargin=*]
    \item Application: Injected
    \item Potency: 80
    \item Resistance: Endurance
    \item Onset time: 1d6+6 days
    \item Duration: 1 week + 1d3 days
    \item Conditions: Disease manifests with the start of Hallucinations. After one week they also start to suffer from Mania (Fire). If the victim somehow survives to the conclusion of the disease, they then suffer Death.
\end{itemize}

\textbf{Red Pox}
\begin{itemize}[leftmargin=*]
    \item Application: Contact
    \item Potency: 50
    \item Resistance: Endurance
    \item Onset time: 1d6+6 hours
    \item Duration: 1d6+3 days
    \item Conditions: Once the disease manifests the victim becomes Contagious. 1 day later they begin to suffer Fever and Bleeding. Each day the pustules weep blood the victim loses one level of Fatigue.
\end{itemize}

\textbf{Soul Leech}
\begin{itemize}[leftmargin=*]
    \item Application: Inhaled
    \item Potency: 65
    \item Resistance: Endurance
    \item Onset time: 1d6 days
    \item Duration: 1d3+3 weeks
    \item Conditions: Sapping. Each week the victim reduces his Magic Points attribute by 1d6 points.
\end{itemize}

\subsection{Sample Poisons}
\label{sec:sample-poisons}
\textbf{Cobra Venom}
\begin{itemize}[leftmargin=*]
    \item Application: Contact (eyes) or Injected
    \item Potency: 75
    \item Resistance: Endurance
    \item Onset time: Instant if spat in eyes, 1d6+4 minutes if bitten
    \item Duration: 1d3+3 days
    \item Conditions: If spat into the eyes both Agony and Blindness are instant. Bitten victims also begin with Agony but can struggle along until 1d6+6 hours after the bite when Asphyxiation strikes. Survivors will then suffer Necrosis, losing 1 Hit Point per day from the location bitten.
\end{itemize}

\textbf{Lotus Dust}
\begin{itemize}[leftmargin=*]
    \item Application: Inhaled
    \item Potency: 90
    \item Resistance: Willpower
    \item Onset time: 1d3 rounds
    \item Duration: Instantaneous in its natural state, 2d6 hours if smoked
    \item Conditions: Death if natural pollen. Paralysis and Hallucinations if smoked.
\end{itemize}

\textbf{Sleeping Draught}
\begin{itemize}[leftmargin=*]
    \item Application: Ingested
    \item Potency: 60
    \item Resistance: Willpower
    \item Onset time: 1d6+4 minutes
    \item Duration: 1d6+3 hours
    \item Conditions: Unconsciousness.
\end{itemize}

\section{ENCUMBRANCE}
\label{sec:encumbrance}
Encumbrance represents both the mass and bulk of an item. Items that have a zero ENC value are, on their own, inconsequential; however consider that 20 zero ENC items equal 1 ENC.

\subsection{Encumbrance Capacity}
\label{sec:enc-capacity}
Characters can carry a total ENC equal to their STR × 2 with relative ease. Everyday clothing does not contribute to this capacity, but armour does.

If the total ENC borne exceeds STR × 2 then the character is considered to be \textbf{Burdened}. This has the following effects:
\begin{itemize}[leftmargin=*]
    \item Skills using STR or DEX as part of their base become one grade harder
    \item Base Movement Rate drops by 2 metres, and the character can no longer sprint
    \item Carrying the load counts as \hyperref[tab:physical-effort]{Medium activity} for Fatigue purposes 
\end{itemize}

If the borne ENC exceeds STR × 3 then the character is considered to be \textbf{Overloaded}:
\begin{itemize}[leftmargin=*]
    \item Skills using STR or DEX as part of their base become two grades harder
    \item Base Movement Rate drops to half normal, and the character cannot move faster than a walk
    \item Carrying the load counts as \hyperref[tab:physical-effort]{Strenuous activity} for Fatigue purposes
\end{itemize}

Characters cannot carry a total ENC more than their STR × 4.

\subsection{Armour ENC}
\label{sec:armor-enc}
The amount armour counts towards Encumbrance Capacity depends on whether it is worn or carried.
\begin{itemize}[leftmargin=*]
    \item When worn, only half the total ENC of the armour is counted towards Encumbrance
    \item When carried, the full ENC value of the armour is counted towards Encumbrance
\end{itemize}


\section{FALLING}
\label{sec:falling}
The amount of damage suffered in a fall depends on the distance of the drop. Armour points do not reduce falling damage.

\begin{table}[h]
\centering
\caption{Falling Distance Table}
\label{tab:falling}
\begin{tabular}{ll}
\toprule
\textbf{Distance Fallen} & \textbf{Damage Taken} \\
\midrule
1m or less & No damage \\
2m to 5m & 1d6 to one random location \\
6m to 10m & 2d6 to two random locations \\
11m to 15m & 3d6 to three random locations \\
16m to 20m & 4d6 to four random locations \\
Each +5m & +1d6 damage \\
\bottomrule
\end{tabular}
\end{table}

Damage is rolled separately for each location; it is not spread among them.

\subsection{Size Modifiers}
\label{sec:falling-size}
Creatures of smaller size suffer less from a fall. Creatures of larger size suffer more from a fall. For every 10 points above SIZ 20 the creature adds +1d6 to the damage.

\begin{table}[h]
\centering
\caption{Size Modifiers for Falling Damage}
\label{tab:falling-size}
\begin{tabular}{ll}
\toprule
\textbf{Creature SIZ} & \textbf{Distance Fallen Adjustment} \\
\midrule
SIZ 8 to 9 & Treat distance as 1 metre less \\
SIZ 6 to 7 & Treat distance as 3 metres less \\
SIZ 4 to 5 & Treat distance as 5 metres less \\
SIZ 2 to 3 & Treat distance as 8 metres less \\
SIZ 1 or less & Treat distance as 10 metres less \\
\midrule
\textbf{Larger Creatures} & \textbf{Damage Adjustment} \\
\midrule
Each 10 points above SIZ 20 & +1d6 damage \\
\bottomrule
\end{tabular}
\end{table}

\subsection{Mitigation}
Acrobatics can be used to mitigate falling damage - a successful roll allows the character to treat the fall as if it were two metres shorter than it actually is. Characters falling onto soft surfaces may treat the distance they fall as halved for the purposes of damage.

\section{FATIGUE}
\label{sec:fatigue}
Fatigue measures tiredness and its incremental effects. The primary way of accruing Fatigue is by engaging in some form of physical activity.

\begin{table}[h]
\centering
\caption{Physical Effort Table}
\label{tab:physical-effort}
\begin{tabular}{lll}
\toprule
\textbf{Effort} & \textbf{How Long?} & \textbf{Skill Roll} \\
\midrule
Light & CON in hours & Very Easy (Athletics/Brawn/Endurance) \\
Medium & CON in minutes & Easy (Athletics/Brawn/Endurance) \\
Strenuous & CON in seconds & Standard (Athletics/Brawn/Endurance) \\
\bottomrule
\end{tabular}
\end{table}

\subsection{Effects of Fatigue}
 Each level of Fatigue carries penalties for skill use, movement, Initiative and Action Points.

\begin{table}[h]
\centering
\caption{Fatigue Levels}
\label{tab:fatigue-levels}
\begin{tabular}{lllll}
\toprule
\textbf{Level} & \textbf{Skill Grade} & \textbf{Move} & \textbf{Init} & \textbf{AP} \\
\midrule
Fresh & No Penalty & — & — & — \\
Winded & Hard & -1m & — & — \\
Tired & Hard & — & -2 & — \\
Wearied & Formidable & -2m & -4 & — \\
Exhausted & Formidable & Halved & -6 & -1 \\
Debilitated & Herculean & Halved & -8 & -2 \\
Incapacitated & Herculean & Immobile & — & -3 \\
Semi-Conscious & Hopeless & \multicolumn{3}{c}{No Activities Possible} \\
Comatose & \multicolumn{4}{c}{No Activities Possible} \\
Dead & \multicolumn{4}{c}{Dead} \\
\bottomrule
\end{tabular}
\end{table}

\subsection{Recovering from Fatigue}
Characters recover from Fatigue depending on their \textcolor{red}{Healing Rate}. The amount of complete rest needed to recover from each level of accrued Fatigue is equal to the \textcolor{red}{Recovery Period} divided by the character's \textcolor{red}{Healing Rate}.

\section{FIRES}
\label{sec:fires}
Fires damage both people and objects; how much is dependent on the intensity of the source.

\begin{table}[h]
\centering
\caption{Fire Intensity Table}
\label{tab:fire}
\begin{tabular}{llll}
\toprule
\textbf{Intensity} & \textbf{Examples} & \textbf{Time to Ignite} & \textbf{Damage} \\
\midrule
1 & Candle & 1d4 & 1d2 \\
2 & Torch & 1d3 & 1d4 \\
3 & Campfire & 1d2 & 1d6 \\
4 & Room conflagration & 1d2 & 2d6 \\
5 & Volcanic lava & Instant & 3d6 \\
\bottomrule
\end{tabular}
\end{table}

\section{HEALING}
\label{sec:healing}
Natural healing from wounds and injuries is based on the character's \textcolor{red}{Healing Rate}. The Healing Rate dictates how many Hit Points are recovered in a location depending on the injury's nature:

\begin{itemize}[leftmargin=*]
    \item \textbf{Minor Wounds:} The character recovers Hit Points at a rate equal to their Healing Rate \textit{per day}.
    \item \textbf{Serious Wounds:} The character recovers Hit Points at a rate equal to their Healing Rate \textit{per week}.
    \item \textbf{Major Wounds:} The character recovers Hit Points at a rate equal to their Healing Rate \textit{per month}.
\end{itemize}



\subsection{Healing Restrictions}
\label{sec:healing-restrictions}
There are certain restrictions on natural healing:
\begin{itemize}[leftmargin=*]
    \item The healing character cannot engage in strenuous activity: otherwise the Healing Rate is reduced by 1d3. Thus, a character recovering from even a Minor Wound could find his progress halted if he decides to engage any physical tasks that might exacerbate his injuries.
    \item Natural healing will not begin to heal a Major Wound until that location has been treated with a successful Healing roll.
    \item Non-dismembering Major Wounds which are not treated with successful Healing roll within a number of days equal to one twentieth of the Healing skill become maimed, permanently reducing the Hit Points of the location.
\end{itemize}

\subsection{Magical Healing}
\label{sec:magic-healing}
Some magic can heal the wounds suffered by a victim. Yet there are specific restrictions as to what level of wound can be treated by each spell. 

No matter how petty the healing spell or miracle, its application is always enough to stabilise any type of wound, preventing bleeding and immediate death even if it doesn't actually cure the underlying injury. Note that this only applies to gross physical trauma, not to conditions brought about by suffocation, poison, and the like.

\subsection{Permanent Injuries}
\label{sec:permanent-injuries}
Some Major Wounds, and certain poisons or diseases inflict maiming injuries. The result of this damage permanently reduces the Hit Points on that location, forever weakening it. A location maimed in this way uses the diminished Hit Point value to calculate its new Serious and Major Wound thresholds.

\begin{table}[h]
\centering
\caption{Permanent Injury Table}
\label{tab:permanent-injury}
\begin{tabular}{cp{10cm}}
\toprule
\textbf{1d3} & \textbf{Maiming Result} \\
\midrule
1 & Character permanently loses one third of the Hit Points in that location. If a limb, this represents the maiming of a hand or foot. If the head, the character loses one of his sensory organs: eye, ear, nose or tongue. Anywhere else it denotes a disfiguring scar. \\
\midrule
2 & Character permanently loses two thirds of the Hit Points in that location. A limb is maimed from the elbow or knee down. The head loses two sensory organs. Torso exhibits a gruesomely horrible scar. \\
\midrule
3 & Location is reduced to a single Hit Point. Limbs are maimed from the shoulder or hip down. The head either loses three sensory organs, half the face or the entire jaw. Chest or abdomen shows such a horrific scar or deformation nobody seeing the healed wound can comprehend how the victim survived. \\
\bottomrule
\end{tabular}
\end{table}

\section{INANIMATE OBJECTS}
\label{sec:objects}
All inanimate objects possess Armour Points and Hit Points which are used to determine resistance to damage and destruction. Armour Points reduce damage before Hit Points are affected. Once an object's Hit Points have been reduced to zero, it is useless.

\begin{table}[h]
\centering
\caption{Inanimate Objects Armour and Hit Points}
\label{tab:objects}
\begin{tabular}{lll}
\toprule
\textbf{Object} & \textbf{Armour Points} & \textbf{Hit Points} \\
\midrule
Boulder & 10 & 40 \\
Castle gate & 8 & 120 \\
Castle wall (2m) & 10 & 250 \\
Chain/shackle & 8 & 8 \\
Club & 4 & 4 \\
Dagger & 6 & 4 \\
Hut wall (2m) & 3 & 15 \\
Iron door & 12 & 75 \\
Rope & 6 & 3 \\
War sword & 6 & 10 \\
Wooden chair & 2 & 6 \\
Wooden door (normal) & 4 & 25 \\
Wooden door (reinforced) & 6 & 30 \\
Wooden fence (2m) & 4 & 5 \\
\bottomrule
\end{tabular}
\end{table}

\section{LUCK POINTS}
\label{sec:luck}
Luck Points help differentiate heroes from the rank and file. Every character starts with a number of Luck Points. Luck Points can be used during play and, at the beginning of the next session, replenish to their usual value.

\subsection{Using Luck Points}
Only one Luck Point can be used in support of a particular action.
\begin{itemize}[leftmargin=*]
    \item \textbf{Cheat Fate:} Re-roll any dice roll that affects them. Force an opponent to re-roll.
    \item \textbf{Desperate Effort:} Gain an additional \textcolor{red}{Action Point} if exhausted.
    \item \textbf{Mitigate Damage:} Suffering a Major \textcolor{red}{Wound} may spend a Luck Point to downgrade the injury to a Serious Wound.
\end{itemize}


\section{PASSIONS}
\label{sec:passions}
A Passion is any deeply held commitment that has the capacity to influence events during play.

\subsection{Using Passions}
\begin{itemize}[leftmargin=*]
    \item To augment another skill, reflecting the depth of one's feeling. Adds 20\% of its value to a skill (if allowed).
    \item As an ability in its own right to drive choices. A standard roll is made against a Passion.
    \item To oppose other Passions - even those held by the same character. This is typically used where two Passions would conflict.
    \item As a general measure of depth of commitment.
    \item To resist psychological manipulation or magical domination, substituting for Willpower (if allowed).
\end{itemize}

\subsection{Deepening and Waning}
Passions can increase during a game independently of \hyperref[sec:improvement]{Experience roll} , based on the strength of whatever occurred to trigger the increase. Passions can also wane.

\begin{table}[h]
\centering
\caption{Deepening and Waning}
\label{tab:passion-change}
\begin{tabular}{ll}
\toprule
\textbf{Change} & \textbf{Amount} \\
\midrule
Weak & 1d10 \\
Moderate & 1d10+5 \\
Strong & 1d10+10 \\
\bottomrule
\end{tabular}
\end{table}

\section{SURVIVAL}
\label{sec:survival}
Characters may find themselves in hostile environments battling the elements: exposure, starvation, and thirst.

Each danger has a specific onset time:
\begin{itemize}[leftmargin=*]
    \item \textbf{Exposure:} Assuming suitable clothing it usually starts to take effect after CON hours.
    \item \textbf{Starvation:} Critical levels begin after a number of days equal to half CON.
    \item \textbf{Dehydration:} Begins after CON x4 hours.
\end{itemize}

Once conditions take hold characters start to accrue \hyperref[sec:fatigue]{Fatigue} levels. An Endurance roll is required at onset, which is then repeated every time the Exposure, Starvation or Dehydration Rate cycles.

\section{TRAPS}
\label{sec:traps}
Traps are described using the following traits.

\subsection{Trap Traits}
\begin{itemize}[leftmargin=*]
    \item \textbf{Purpose:} Alarm, Ensnaring, Maiming, Death.
    \item \textbf{Trigger:} The way the trap is set off.
    \item \textbf{Difficulty:} The challenge rating of the trap, equal to the value of the Mechanisms or Engineering skill which created it.
    \item \textbf{Resistance:} How the trap is resisted (Brawn, Evade, or parrying with a Combat Style).
    \item \textbf{Effect:} What happens when the trap is sprung.
\end{itemize}

\begin{table}[h]
\centering
\caption{Death Trap Damage Limits}
\label{tab:trap-damage}
\begin{tabular}{lll}
\toprule
\textbf{Maker's Skill} & \textbf{Damage} & \textbf{Size/Force} \\
\midrule
1-10\% & 1d2 & Small \\
11-20\% & 1d4 & Small \\
21-30\% & 1d6 & Medium \\
31-40\% & 1d8 & Medium \\
41-50\% & 1d10 & Large \\
51-60\% & 2d6 & Large \\
61-70\% & 1d8+1d6 & Huge \\
71-80\% & 2d8 & Huge \\
81-90\% & 1d10+1d8 & Enormous \\
91-100\% & 2d10 & Enormous \\
\bottomrule
\end{tabular}
\end{table}

\subsection{Sample Traps}
\label{sec:sample-traps}
\textbf{Crushing Roof}
\begin{itemize}[leftmargin=*]
    \item Purpose: Death
    \item Difficulty: 80\%
    \item Resistance: Evade to dive clear. If trapped, combined Brawn vs 2d8.
    \item Effect: Once the ceiling lowers far enough, victims become pinned to the floor, and receive 2d8 damage per round to a random Hit Location (armour does not protect).
\end{itemize}

\textbf{Pitfall}
\begin{itemize}[leftmargin=*]
    \item Purpose: Ensnaring or Death
    \item Difficulty: 60\%
    \item Resistance: Evade to jump clear or a Hard Athletics roll to catch the edge.
    \item Effect: The drop inflicts 2d6 damage to a random Hit Location, armour does not protect.
\end{itemize}

\textbf{Spear Trap}
\begin{itemize}[leftmargin=*]
    \item Purpose: Maiming
    \item Difficulty: 75\%
    \item Resistance: Evade to dive aside or a Hard parry roll if wielding a shield.
    \item Effect: The spear inflicts 2d8 damage to a random Hit Location.
\end{itemize}

\section{VISIBILITY}
\label{sec:visibility}
The table below gives rough distances in metres for the visibility of man-sized objects, according to the ambient weather and quality of light.

The SIZ of what someone is trying to observe also affects its visibility, as will other factors such as obscuring undergrowth, background terrain, and possible camouflage. Assume targets with SIZ 10 or below halve the range at which they can be spotted, and that larger objects increase the range by one multiple for every 10 points of SIZ over 20.

\begin{table}[h]
\centering
\caption{Visibility Table (meters for man-sized objects)}
\label{tab:visibility}
\begin{tabular}{lcccc}
\toprule
\textbf{Conditions} & \textbf{Daylight} & \textbf{Twilight} & \textbf{Moonlight} & \textbf{Moonless Night} \\
\midrule
Clear & 500 & 300 & 50 & 25 \\
Overcast & 400 & 250 & 50 & 25 \\
Moderate Fog & 150 & 100 & 25 & 15 \\
Dense Fog & 50 & 30 & 15 & 10 \\
Wind-driven snow & 35 & 25 & 10 & 5 \\
Sand/dust storm & 20 & 15 & 10 & 5 \\
\bottomrule
\end{tabular}
\end{table}

\section{WEATHER}
\label{sec:weather}
Weather conditions can have a significant effect on the local environment. The main aspects of weather are precipitation, temperature, and wind. 

\subsection{Temperature}
\label{sec:temperature}
Temperature can vary dramatically due to climate, season, and elevation, depending on the campaign world and where scenarios are set. The following table provides a guideline for the effects of extreme temperature on characters.

Wearing suitable local clothing permits a character a grace period of their CON in hours before exposure sets in. After this point they need to seek shelter or start to suffer \hyperref[sec:fatigue]{Fatigue} loss. Wet characters shift the Exposure Rate one step cooler. Light or moderate gale force winds also shift Exposure Rate by one step cooler, whilst strong gales and storms shift it two steps.

\begin{table}[h]
\centering
\caption{Temperature Table}
\label{tab:temperature}
\begin{tabular}{llll}
\toprule
\textbf{Temp °C} & \textbf{Climate} & \textbf{Risk of Exposure} & \textbf{Exposure Rate} \\
\midrule
Below -20 & Glacial & Yes & Minutes \\
-19 – -10 & Freezing & Yes & 15 Minutes \\
-9 – 0 & Cold & Yes & Hourly \\
1 – 10 & Chill & Yes & Daily \\
11 – 20 & Cool & No & None \\
21 – 30 & Warm & No & None \\
31 – 40 & Hot & Yes & Hourly \\
\bottomrule
\end{tabular}
\end{table}

\subsection{Precipitation}
\label{sec:precipitation}
Precipitation relates to the amount of moisture which falls out of the sky. Depending on the temperature it can range between rain, sleet and snow, with hail falling during storms. The base chance of it raining is equal to the relative humidity. The amount of rain per hour and duration of the fall can be calculated by looking up the relevant entry on the Precipitation table.

Normally precipitation has little effect on characters save to slow down travel if excessive rain causes flooding, or snow begins to drift, obscuring or blocking paths.

\begin{table}[h]
\centering
\caption{Precipitation Table}
\label{tab:precipitation}
\begin{tabular}{lllll}
\toprule
\textbf{Rel. Humidity} & \textbf{Typical Cloud Cover} & \textbf{Amount per Hour} & \textbf{Duration} & \textbf{Dehydration Rate} \\
\midrule
0-12\% & None & None & None & Hourly \\
13-25\% & Scant cloud & Very light (0-1mm) & 1d10 minutes & 2 Hours \\
26-37\% & Scattered cloud & Light (1-2.5mm) & 1d6 ×10 minutes & 3 Hours \\
38-50\% & Heavy cloud & Moderate (2.5-10mm) & 1d2 hours & 4 Hours \\
51-62\% & Slightly Overcast & Heavy (11-25mm) & 1d3 hours & 4 Hours \\
63-75\% & Moderately Overcast & Very Heavy (26-50mm) & 1d6 hours & 3 Hours \\
76-87\% & Completely Overcast & Monsoon (51-80mm) & 1d8 hours & 2 Hours \\
88-100\% & Storm Clouds & Deluge (81+mm) & 1d12 hours & Hourly \\
\bottomrule
\end{tabular}
\end{table}

\subsection{Wind}
\label{sec:wind}
Wind, especially very strong winds, can have an adverse effect on activity. A wind's Strength (STR) is expressed in kilometres per hour of velocity. Its effect on physical skills – those involving STR or DEX – is detailed in the Skill Grade column; effects on Movement Rate in the Movement Rate column.

\begin{table}[h]
\centering
\caption{Wind Table}
\label{tab:wind}
\begin{tabular}{llll}
\toprule
\textbf{Wind STR} & \textbf{Name} & \textbf{Skill Grade} & \textbf{Movement Rate} \\
\midrule
0-15 & Calm Day/Light Breeze & Standard & Normal \\
16-30 & Moderate Breeze & Standard & Normal \\
31-45 & Strong Breeze & Standard & Two Thirds \\
46-60 & Light Gale & Hard & Two Thirds \\
61-75 & Moderate Gale & Hard & Half \\
76-90 & Strong Gale & Formidable & Half \\
91+ & Storm/Hurricane & Formidable & One Third \\
\bottomrule
\end{tabular}
\end{table}

\end{document}
